\documentclass[11pt,addpoints,letterpaper]{exam}
\usepackage{amsmath,tcolorbox,mdframed,xcolor}
\usepackage[top=0.75in, bottom=0.85in, left=1in, right=1in]{geometry} % Adjust margins

\usepackage{siunitx}
\usepackage{nccmath}
\usepackage{physics}
\usepackage{tikz}
\usepackage{multicol}
\usepackage{xfrac}
\usepackage{wrapfig}

\usepackage{float}


\noprintanswers


% Define variables
\newcommand{\instructor}{Mr. Rodriguez}
\newcommand{\coursename}{Conceptual Physics A}
\newcommand{\term}{Fall}
\newcommand{\courseyear}{2024} % Renamed \year to \courseyear to avoid conflict
\newcommand{\instructions}{}
\newcommand{\worksheetname}{Final Exam Review}
\newcommand{\qsp}{\vspace{5cm}}
\newcommand{\qspp}{\vspace{2.5cm}}
\newcommand{\qsppp}{\vspace{0.5cm}}

\tcbset{
    myboxstyle/.style={
        colback=white,        % Background color
        colframe=black,       % Border color
        boxrule=0.5pt,        % Border thickness
        arc=5mm,              % Rounding radius
        boxsep=2mm,           % Padding around text
        left=4pt, right=4pt,  % Inner padding on left and right
    }
}
\newcommand{\equationbox}[2]{
\begin{center}
\begin{tcolorbox}[colframe=black!60, colback=white, arc=5mm, boxrule=0.75pt, title=#1]
% \vspace{-0.2in} % Tightens the space between the title and content
\begin{flushleft}
\begin{align*}
#2
\end{align*}
\end{flushleft}
\end{tcolorbox}
\end{center}
}
\newcommand{\equationboxx}[2]{
\begin{center}
\begin{tcolorbox}[colframe=black!60, colback=white, arc=5mm, boxrule=0.75pt, title=#1]
\begin{flushleft}
#2
\end{flushleft}
\end{tcolorbox}
\end{center}
}
\newcommand{\instructionbox}[1]{\begin{center}\fbox{\fbox{{\centering #1}}}\end{center}}

\newcommand{\learningStandard}[1]{
    \small
    \tcbox[colback=gray!10, colframe=black, boxrule=0.5mm, sharp corners=southwest]{
        \textbf{#1} \hspace{0.5cm} % Spacing between title and vertical rule
        \textcolor{gray}{\vrule height 10pt width 0.3pt}\hspace{0.5cm} % Lighter and thinner vertical rule
        \textbf{Score:} \rule{1.5cm}{0.5pt} /10
    }
}

\newtcolorbox{learningStandardBox}[2]{%
    title={Learning Standard #1}, % Title with manual input for the standard number
    fonttitle=\bfseries\small,  % Title font style
    colback=gray!15, % Background color for main content
    colframe=black, % Frame color
    coltitle=black!80, % Font color for title
    colbacktitle=gray!60, % Background color for title
    boxrule=0.5mm, % Border thickness
    sharp corners=south, % Rounded top corners only
    width=\textwidth, % Full text width
    halign=center, % Center-align content
    top=10pt, % Adds extra space at the top to move content down
}
\newcommand{\standardBox}[2]{%
    \begin{learningStandardBox}{#1}{#2} % Passes number and name as arguments
        \begin{minipage}[t]{0.5\textwidth} % Left section for standard name
            \textit{#2} % Italicized name of the standard
        \end{minipage}%
        \hfill
        \begin{minipage}[t]{0.25\textwidth} % Middle section for score
            % \textbf{} \rule{1.5cm}{0.5pt} /10
        \end{minipage}%
        \hfill
        \begin{minipage}[t]{0.2\textwidth} % Right section for grade
            \textbf{} \phantom{1.2cm}
        \end{minipage}
    \end{learningStandardBox}
}
%%%%%%%%%%%%%%%%%%%%%% BEGIN DOCUMENT %%%%%%%%%%%%%%%%%%%%%%%%%%%%%%%%%%%%%%%
\begin{document}

% Place name, section, and instructor details at the top left
\vspace*{-0.2in} % Adjust this to control vertical spacing from the top of the page
\noindent
\makebox[0.7\textwidth][l]{Name:\enspace\hrulefill} \hfill \makebox[0.3\textwidth]{ Period:\enspace\hrulefill} \\[1cm]
\noindent
\makebox[0.7\textwidth][l]{Instructor: \enspace\texttt{\instructor}}\hfill \hspace*{0pt}\hfill{Total Score:\underline{\hspace{2cm}}/\numpoints}
\vspace{0.5in} % Space before the title


% Manually add the title without using \maketitle
\begin{centering}
\noindent\textbf{\Large \worksheetname} \\[0.1in]
\noindent\texttt{\coursename} \\
\noindent\textit{\term\ \courseyear} \\
\end{centering}

\vspace{0.2in}

% Special Instructions
% \begin{center}
% \fbox{\fbox{\parbox{5.5in}{\centering
% \instructions}}}
% \end{center}


% Relevant Equations box with no extra space and left-aligned equations


%%%%%%%%%%%%%%%%%%%%%%%%%%%%%%%%%%%%%%%%%%% LS 1 %%%%%%%%%%%%%%%%%%%%%%%%%%%%%%%%%%%%%%
% \section*{\small \tcbox{Learning Standard 3.1: The Law of Conservation of Energy}}
% \section*{\learningStandard{Learning Standard 3.1: The Law of Conservation of Energy}}
\standardBox{3.1}{The Law of Conservation of Energy}


\subsection*{Key Concepts}
\begin{itemize}
    \item Understand the law of conservation of energy: energy cannot be created or destroyed, only transferred or transformed.
    \item Know the SI units of energy (Joules, J).
\end{itemize}

\subsection*{Practice}
\begin{itemize}
    \item Define and explain the law of conservation of energy.
    \item Identify different types of energy (kinetic, potential, chemical, etc.) and how energy is converted in systems like:
    \begin{itemize}
        \item Speakers
        \item Solar panels
        \item Engines
        \item Flashlights
        \item Bow and arrow
        \item Turbines
    \end{itemize}
    \item Example question: How does a flashlight convert energy?
\end{itemize}

\subsection*{Application}
Relate energy transformations on Earth to the Sun as the primary energy source and how humans, animals, plants, and fossil fules play into that network. 


\pagebreak

\standardBox{3.2}{Problem Solving via Conservation of Energy}
\subsection*{Key Concepts}
\begin{itemize}
    \item Use the conservation of energy principle (e.g. potential energy turning into kinetic energy or vice versa) to solve problems.
    \item Understand energy relationships in motion, for example the kid on the swing problem from class.
\end{itemize}

\subsection*{Practice}
\begin{itemize}
    \item Derive the minimum maximum height that a roller skater will go up a ramp given that their starting speed is $v=\SI{10}{m/s}$ and neglecting friction and air resistance.
    \item Work through problems involving:
    \begin{itemize}
        \item Gravitational potential energy ($PE = mgh$)
        \item Kinetic energy ($KE = \frac{1}{2}mv^2$)
    \end{itemize}
\end{itemize}

\subsection*{Application}
Interpret how increasing speed or energy affects motion, especially in systems where energy transforms from potential to kinetic or vice versa.

\pagebreak
\standardBox{3.3}{Power and Generators}
\subsection*{Key Concepts}
\begin{itemize}
    \item Understand power as the rate of energy transfer ($P = \frac{\Delta E}{\Delta t}$).
    \item Calculate potential energy and power for systems like hydroelectric turbines.
\end{itemize}

\subsection*{Practice}
\begin{itemize}
    \item Calculate gravitational potential energy for falling water (e.g., Niagra Falls) using $PE = mgh$.
    \item Determine the theoretical power output of water flow and relate it to powering devices (e.g., number of $20\,\mathrm{W}$ light bulbs).
\end{itemize}

\subsection*{Application}
Solve real-world problems involving power generation and energy efficiency. Example problem: The water at Niagara Falls flows from a height of \SI{50}{m} and \SI{5.0e5}{kg} of water flows over the falls every second. Assume 85 percent of the gravitational potential energy of the water is converted into electrical power by a hydroelectric turbine. What is the usable power of this system in Watts?



\pagebreak
\standardBox{3.4}{Energy and Automobile Safety}
\subsection*{Key Concepts}
\begin{itemize}
    \item Understand kinetic energy and how it relates to velocity ($KE = \frac{1}{2}mv^2$).
    \item Analyze work and force in stopping a car ($W = Fd$).
\end{itemize}

\subsection*{Practice}
\begin{itemize}
    \item Calculate the initial kinetic energy of a car given mass and velocity.
    \item Determine stopping distance when a constant force is applied.
    \item Explore the effects of changing the speed on kinetic energy and stopping distance.
\end{itemize}

\subsection*{Application}
Discuss the relationship between speed, kinetic energy, and crash severity. Example question: If speed triples, by what factor does stopping distance increase?




\end{document}
