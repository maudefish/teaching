\documentclass[11pt,addpoints,letterpaper]{exam}
\usepackage{adjustbox} % Add this to the preamble
\usepackage{float}

\usepackage{amsmath,tcolorbox,mdframed,xcolor}
\usepackage[top=0.75in, bottom=0.85in, left=1in, right=1in]{geometry} % Adjust margins
\usepackage{mathtools}
\usepackage{siunitx}
\usepackage{nccmath}
\usepackage{physics}
\usepackage{tikz}
\usepackage{multicol}
\usepackage{epigraph}
\usepackage{cancel}
\usepackage{xfrac}
\usepackage{pgfplots}
\usepackage{array}   % For alignment
\usepackage{booktabs} % For better horizontal lines
\usepackage{caption} % For caption formatting
\usepackage{graphicx} % For scaling tables if needed
\usepackage{setspace}

\setlength\epigraphwidth{.8\textwidth}
\setlength\epigraphrule{0pt}

\noprintanswers

\newcommand{\Section}[1]{
\textbf{\\\Large #1\\}
}
\newcommand{\labheader}[4]{%
  \vspace*{-0.2in} % Adjust vertical spacing
  \noindent
  \makebox[0.65\textwidth][l]{Name:\enspace\hrulefill\,\,\,}
  \makebox[0.1\textwidth][l]{\,Period:\enspace}
  \makebox[0.25\textwidth][l]{\hrulefill}\\[0.25cm]
  \makebox[0.65\textwidth][l]{Instructor: \enspace\texttt{\underline{#1}}}\hfill
  \makebox[0.1\textwidth][l]{ Course: }\hfill 
  \makebox[0.25\textwidth][l]{ \texttt{\underline{#2}}}\hfill \\[0.25cm]
  \makebox[0.65\textwidth]{}\makebox[0.1\textwidth][l]{ Term: }\hfill
  \makebox[0.25\textwidth][l]{\enspace\texttt{\underline{#3}}}\hfill
  % \hfill{Score: \underline{\hspace{2cm}}/#4}
  \vspace{0.5in}
}

% Define variables
\newcommand{\instructor}{Mr. Rodriguez}
\newcommand{\coursename}{Conceptual Physics A}
\newcommand{\term}{Winter}
\newcommand{\courseyear}{2024-2025} % Renamed \year to \courseyear to avoid conflict
\newcommand{\instructions}{}
\newcommand{\worksheetname}{Lab 4: Inertia and Oscillations}
\newcommand{\qspppp}{\vspace{6cm}}
\newcommand{\qsppp}{\vspace{4.5cm}}
\newcommand{\qspp}{\vspace{2.5cm}}
\newcommand{\qsp}{\vspace{0.6cm}}

\tcbset{
    myboxstyle/.style={
        colback=white,        % Background color
        colframe=black,       % Border color
        boxrule=0.5pt,        % Border thickness
        arc=5mm,              % Rounding radius
        boxsep=2mm,           % Padding around text
        left=4pt, right=4pt,  % Inner padding on left and right
    }
}
\newcommand{\equationbox}[2]{
\begin{center}
\begin{tcolorbox}[colframe=black!60, colback=white, arc=5mm, boxrule=0.75pt, title=#1]
% \vspace{-0.2in} % Tightens the space between the title and content
\begin{flushleft}
\begin{align}
#2
\end{align}
\end{flushleft}
\end{tcolorbox}
\end{center}
}
\newcommand{\equationboxx}[2]{
\begin{center}
\begin{tcolorbox}[colframe=black!60, colback=white, arc=5mm, boxrule=0.75pt, title=#1]

#2

\end{tcolorbox}
\end{center}
}
\newcommand{\instructionbox}[1]{\begin{center}\fbox{\fbox{{\centering #1}}}\end{center}}

\newcommand{\learningStandard}[1]{
    \small
    \tcbox[colback=gray!10, colframe=black, boxrule=0.5mm, sharp corners=southwest]{
        \textbf{#1} \hspace{0.5cm} % Spacing between title and vertical rule
        \textcolor{gray}{\vrule height 10pt width 0.3pt}\hspace{0.5cm} % Lighter and thinner vertical rule
        \textbf{Score:} \rule{1.5cm}{0.5pt} /10
    }
}

\newtcolorbox{learningStandardBox}[2]{%
    title={Learning Standard #1}, % Title with manual input for the standard number
    fonttitle=\bfseries\small,  % Title font style
    colback=gray!15, % Background color for main content
    colframe=black, % Frame color
    coltitle=black!80, % Font color for title
    colbacktitle=gray!60, % Background color for title
    boxrule=0.5mm, % Border thickness
    sharp corners=south, % Rounded top corners only
    width=\textwidth, % Full text width
    halign=center, % Center-align content
    top=10pt, % Adds extra space at the top to move content down
}
\newcommand{\standardBox}[2]{%
    \begin{learningStandardBox}{#1}{#2} % Passes number and name as arguments
        \begin{minipage}[t]{0.5\textwidth} % Left section for standard name
            \textit{#2} % Italicized name of the standard
        \end{minipage}%
        \hfill
        \begin{minipage}[t]{0.25\textwidth} % Middle section for score
            \textbf{Score:} \rule{1.5cm}{0.5pt} /10
        \end{minipage}%
        \hfill
        \begin{minipage}[t]{0.2\textwidth} % Right section for grade
            \textbf{Grade:} \phantom{1.2cm}
        \end{minipage}
    \end{learningStandardBox}
}
\newcommand{\standardBoxnoscore}[2]{%
    \begin{learningStandardBox}{#1}{#2} % Passes number and name as arguments
        \begin{minipage}[t]{0.5\textwidth} % Left section for standard name
            \textit{#2} % Italicized name of the standard
        \end{minipage}%
        \hfill
        \begin{minipage}[t]{0.25\textwidth} % Middle section for score
            \textbf{Score:} \rule{1.5cm}{0.5pt} /0
        \end{minipage}%
        \hfill
        \begin{minipage}[t]{0.2\textwidth} % Right section for grade
            \textbf{Grade: N/A} \phantom{1.2cm}
        \end{minipage}
    \end{learningStandardBox}
}


\newcommand{\exampleBox}[2]{%
    \begin{tcolorbox}[title=\texttt{Example}]
        \texttt{Problem:} {#1\\} 
        \texttt{Solution:} {#2}
    \end{tcolorbox}%
}

\newcommand{\customsection}[1]{%
    \vspace{1em} % Add some vertical space above (optional)
    {\Large\bfseries #1} % Format the title: large font and bold
    \par\vspace{0.5em}\noindent % Add some space below
}

%%%%%%%%%%%%%%%%%%%%%% BEGIN DOCUMENT %%%%%%%%%%%%%%%%%%%%%%%%%%%%%%%%%%%%%%%
\begin{document}

% Place name, section, and instructor details at the top left
\vspace*{-0.2in} % Adjust this to control vertical spacing from the top of the page
\noindent
\makebox[0.65\textwidth][l]{Name:\enspace\hrulefill\,\,\,} 
\makebox[0.1\textwidth][l]{\,Period:\enspace} 
\makebox[0.25\textwidth][l]{\hrulefill}\\[0.25cm]
\makebox[0.65\textwidth][l]{Instructor: \enspace\texttt{\underline\instructor}}\hfill 
\makebox[0.1\textwidth][l]{ Course: }\hfill
\makebox[0.25\textwidth][l]{ \texttt{\underline\coursename}}\hfill \\[0.25cm]
\makebox[0.65\textwidth]{}\makebox[0.1\textwidth][l]{ Term: }\hfill
\makebox[0.25\textwidth][l]{\enspace\texttt{\underline{\term\ \courseyear}}}\hfill 
\vspace{0.5in} % Space before the title

\begin{centering}
\noindent\textbf{\Large \worksheetname} \\[0.1in]
\end{centering}
\vspace{0.1in}
\epigraph{\itshape``Mathematics is the alphabet with which God has written the universe.''}{---Galileo Galilei}
\qsp
% Math is the language of physics. Although the scope of this course will be largely \emph{qualitative} (\emph{i.e.} conceptual), we will need to have a certain level of mathematical fluency under our belts in order to describe the universe around us. Physics is by nature a \emph{quantitative} science--- in other words: numbers are necessary to accurately and precisely depict the world around us. 
\section{Single Variable Equations}
In your algebra course, you have learned how to solve equations for a particular variable. For example, you have undoubtedly encountered many problems of the form,  
\qsp

\exampleBox{
\emph{Solve the following equation for $x$:} 
\begin{align*}
2x + 6 = 10. 
\end{align*}
}{How do we do it? Well, we may either stare at the equation long enough and guess the solution (Think: \emph{``What number when doubled and added to six gives ten?''}), or we may fall back and the tried and true algebraic manipulations you have mastered in your math courses. The fundamental rule in solving any equation is, as always:
\begin{enumerate}
\item \textbf{All mathematical operations must be applied to both sides of the equation.}
\end{enumerate}
To wit: 
\begin{align*}
2x + 6 &= 10 & & \text{(write down the problem)} \\
2x &= 4 & & \text{(subtract 6 from both sides)} \\
\Aboxed{x &= 2.} & & \text{(divide both sides by 2)} 
\end{align*}
It is always good practice to box your final answers.}
\qsp
Now give it a try yourself:
\begin{questions}

\question Solve $2x-4 = 2$ for $x$. 
\ifprintanswers
\begin{solution}
\begin{align*}
2x - 4 &= 2 & & \text{(write down the problem)} \\
2x &= 6 & & \text{(add 4 to both sides)} \\
\Aboxed{x &= 3.} & & \text{(divide both sides by 2)}
\end{align*}
\end{solution}
\else
\qsppp
\fi

\question Solve $4x - 9 = 7$ for $x$.
\ifprintanswers
\begin{solution}
\begin{align*}
4x - 9 &= 7 & & \text{(write down the problem)} \\
4x &= 16 & & \text{(add 9 to both sides)} \\
\Aboxed{x &= 4.} & & \text{(divide both sides by 4)}
\end{align*}
\end{solution}
\else
\qsppp
\fi

\question Solve $6x - 3 = 3x + 9$ for $x$.
\ifprintanswers
\begin{solution}
\begin{align*}
6x - 3 &= 3x + 9 & & \text{(write down the problem)} \\
3x - 3 &= 9 & & \text{(subtract $3x$ from both sides)} \\
3x &= 12 & & \text{(add 3 to both sides)} \\
\Aboxed{x &= 4.} & & \text{(divide both sides by 3)}
\end{align*}
\end{solution}
\else
\qsppp
\fi
\question Solve \( 3y + 2 = 14 \) for \( y \).
\ifprintanswers
\begin{solution}
\begin{align*}
3y + 2 &= 14 & & \text{(write down the problem)} \\
3y &= 12 & & \text{(subtract 2 from both sides)} \\
\Aboxed{y &= 4.} & & \text{(divide both sides by 3)}
\end{align*}
\end{solution}
\else
\qsppp
\fi

\question Solve \( 5z - 8 = 27 \) for \( z \).
\ifprintanswers
\begin{solution}
\begin{align*}
5z - 8 &= 27 & & \text{(write down the problem)} \\
5z &= 35 & & \text{(add 8 to both sides)} \\
\Aboxed{z &= 7.} & & \text{(divide both sides by 5)}
\end{align*}
\end{solution}
\else
\qsppp
\fi

\end{questions}
\newpage
\section{Double Variable Equations}
Oftentimes in physics problems, there is more than one variable involved. If there are two --- call them $x$ and $y$ --- we might refer to them as \emph{independent} and \emph{dependent} variables, respectively. The naming convention reminds us that $y$ depends on $x$; if we change $x$, we expect $y$ to change as well. 

In a science experiment, for example, $x$ might represent the amount of water given each day to a plant in mL (milliliters), while $y$ might represent the height of the plant in cm (centimeters). 
\qsp
\exampleBox{
\emph{Solve for $y$. Then interpret the physical meaning of the equation:} 
\begin{align*}
2y - 10x &= 20. 
\end{align*}
}{
\begin{align*}
2y - 10x &= 20 & & \text{(write down the problem)} \\
2y &= 10x + 20 & & \text{(add \( 10x \) to both sides)} \\
\Aboxed{y &= 5x + 10.} & & \text{(divide both sides by 2)}
\end{align*}
Because the equation takes the familiar $y=mx + b$ form, we could say that ``$y$ increases \textbf{linearly} with $x$.'' In the context of the scientific situation described above the example box, this equation might then be interpreted to be 
\begin{align*}
\Aboxed{\text{(height of plant in cm)} &= 5\times\text{(water given to plant in mL)} + 10.} 
\end{align*}
If we knew that a particular plant was given $x=\SI{5}{mL}$ of water, say, we could plug this into our equation to find the expected height of the plant: 
\begin{align*}
y &= 5(5) + 10 & & \text{(plug in \( x = 5 \))} \\
y &= 25 + 10 & & \text{(add the numbers)} \\
\Aboxed{y &= \SI{35}{cm} } & & \text{(the expected height of the plant is \( \SI{35}{cm} \))}
\end{align*}
}
\begin{questions}
\question 
\begin{parts}
\part Solve $2x + 3y = 12$ for $y$. 
\qspp
\part If $x=4$, what is $y$?
\qspp
\end{parts}
\qsppp
\question
\begin{parts}
\part Solve $5a - 2b = 20$ for $b$. 
\qspp
\part If $a = 6$, what is $b$?
\qspp
\end{parts}
\question
\begin{parts}
\part Solve $\frac{1}{f} = g$ for $f$. 
\qspp
\part If $g = 2$, what is $f$?
\qspp
\end{parts}
\question 
\begin{parts}
\part Say that you are (quickly) walking to class. Every 1 second, you walk 2 meters. Could you write an equation that gives you the distance $d$ that you walk in terms of the seconds that you are walking (call the number of seconds $t$)?
\qsppp
\part Using your equation from the previous part, calculate far would you be able to walk in 60 seconds.
\qsppp
\end{parts}
\end{questions}

\newpage
\section{Full-Variable Equations}
It is also possible (and quite common) for physical equations to contain more than two variables. For example, say that the average speed (velocity) of your car $v$ is equal to the total distance $d$ traveled divided by the time $t$ of travel. The equation that describes this situation would be 
\begin{align*}
v &= \frac{d}{t}.
\end{align*}
If you were to travel $d=\SI{100}{mi}$ in $t=\SI{2}{hr}$, your average speed would be 
\begin{align*}
v &= \frac{\SI{100}{mi}}{\SI{2}{hr}} = \SI{50}{mph}.
\end{align*}
\exampleBox{
Say that you are on the freeway going $v=\SI{65}{mph}$. How far will you travel in $\SI{0.5}{hr}$?
}
{
To solve this problem, we need to rearrange the velocity equation $v = \sfrac{d}{t}$ for our desired variable, namely the distance $d$:
\begin{align*}
v &= \frac{d}{t} & & \text{(start with the given equation)} \\
t \times v &= \frac{d}{t} \times t & & \text{(multiply both sides by t)} \\
t \times v &= \frac{d}{\bcancel{t}} \times \bcancel{t} & & \text{(cancel the t's)} \\
d &= v\,t & & \text{(swap left and right sides of the equation)} \\
d &= (\SI{65}{mph})(\SI{0.5}{hr}) & & \text{(plug in the given values)} \\
\Aboxed{d &= \SI{32.5}{mi} }& & \text{(box final answer)}
\end{align*}
}
\begin{questions}
\question Given the equation $F=ma$,
\begin{parts}
\part Solve for $m$ in terms of $F$ and $a$. Your answer will be a fraction.
\qsppp
\part Solve for $a$ in terms of $F$ and $m$. Your answer again will be a fraction. 
\qsppp
\end{parts}
\question
\begin{parts}
\part Solve the velocity equation $v=d/t$ for the time $t$.
\qsppp
\part Use your result from the previous part to find how long would it take you to travel $d=\SI{200}{mi}$ if you are going $v=\SI{65}{mph}$. 
\qsppp
\end{parts}

\question The speed of light can be written as $c=\lambda f$, where $c$ is the speed of light, $\lambda$ is the wavelength of light, and $f$ is the frequency of the light. (By the way, the Greek letter $\lambda$ is pronounced ``lamb-dah''.)
\begin{parts}
\part Solve for the frequency $f$ in terms of $c$ and $\lambda$.
\qsppp
\part If the speed of light is $c=\SI{300000000}{m/s}$ and the wavelength of a beam of light is $\lambda=\SI{10}{m}$, what is the frequency $f$ of the beam of light?
\qsppp
\end{parts}
\question \emph{Challenge Problem:} Consider the equation
\begin{align*}
\frac{F}{A} = P + \frac{B}{C}.
\end{align*}
How would you go about solving for $C$ in terms of the other 4 variables?
\qsppp
\question \emph{Challenge Problem:} Consider the following two equations:
\begin{align*}
E =& hf & & c = \lambda f. 
\end{align*}
The first equation represents the energy of a packet of light in terms of its frequency and a constant $h$. Use both equations to solve for $E$ in terms of $h$, $c$, and $\lambda$. 
\end{questions}
\end{document}
