\documentclass[11pt,addpoints,letterpaper]{exam}
\usepackage{amsmath,tcolorbox,mdframed,xcolor}
\usepackage[top=0.75in, bottom=0.85in, left=1in, right=1in]{geometry} % Adjust margins
\usepackage{mathtools}

\usepackage{siunitx}
\usepackage{nccmath}
\usepackage{physics}
\usepackage{tikz}
\usepackage{multicol}
\usepackage{epigraph}
\setlength\epigraphwidth{.8\textwidth}
\setlength\epigraphrule{0pt}

\noprintanswers


% Define variables
\newcommand{\instructor}{Mr. Rodriguez}
\newcommand{\coursename}{Conceptual Physics A}
\newcommand{\term}{Winter}
\newcommand{\courseyear}{2024} % Renamed \year to \courseyear to avoid conflict
\newcommand{\instructions}{}
\newcommand{\worksheetname}{Math Practice 1}
\newcommand{\qsppp}{\vspace{4cm}}
\newcommand{\qspp}{\vspace{2.5cm}}
\newcommand{\qsp}{\vspace{0.6cm}}

\tcbset{
    myboxstyle/.style={
        colback=white,        % Background color
        colframe=black,       % Border color
        boxrule=0.5pt,        % Border thickness
        arc=5mm,              % Rounding radius
        boxsep=2mm,           % Padding around text
        left=4pt, right=4pt,  % Inner padding on left and right
    }
}
\newcommand{\equationbox}[2]{
\begin{center}
\begin{tcolorbox}[colframe=black!60, colback=white, arc=5mm, boxrule=0.75pt, title=#1]
% \vspace{-0.2in} % Tightens the space between the title and content
\begin{flushleft}
\begin{align}
#2
\end{align}
\end{flushleft}
\end{tcolorbox}
\end{center}
}
\newcommand{\equationboxx}[2]{
\begin{center}
\begin{tcolorbox}[colframe=black!60, colback=white, arc=5mm, boxrule=0.75pt, title=#1]
\begin{flushleft}
#2
\end{flushleft}
\end{tcolorbox}
\end{center}
}
\newcommand{\instructionbox}[1]{\begin{center}\fbox{\fbox{{\centering #1}}}\end{center}}

\newcommand{\learningStandard}[1]{
    \small
    \tcbox[colback=gray!10, colframe=black, boxrule=0.5mm, sharp corners=southwest]{
        \textbf{#1} \hspace{0.5cm} % Spacing between title and vertical rule
        \textcolor{gray}{\vrule height 10pt width 0.3pt}\hspace{0.5cm} % Lighter and thinner vertical rule
        \textbf{Score:} \rule{1.5cm}{0.5pt} /10
    }
}

\newtcolorbox{learningStandardBox}[2]{%
    title={Learning Standard #1}, % Title with manual input for the standard number
    fonttitle=\bfseries\small,  % Title font style
    colback=gray!15, % Background color for main content
    colframe=black, % Frame color
    coltitle=black!80, % Font color for title
    colbacktitle=gray!60, % Background color for title
    boxrule=0.5mm, % Border thickness
    sharp corners=south, % Rounded top corners only
    width=\textwidth, % Full text width
    halign=center, % Center-align content
    top=10pt, % Adds extra space at the top to move content down
}
\newcommand{\standardBox}[2]{%
    \begin{learningStandardBox}{#1}{#2} % Passes number and name as arguments
        \begin{minipage}[t]{0.5\textwidth} % Left section for standard name
            \textit{#2} % Italicized name of the standard
        \end{minipage}%
        \hfill
        \begin{minipage}[t]{0.25\textwidth} % Middle section for score
            \textbf{Score:} \rule{1.5cm}{0.5pt} /10
        \end{minipage}%
        \hfill
        \begin{minipage}[t]{0.2\textwidth} % Right section for grade
            \textbf{Grade:} \phantom{1.2cm}
        \end{minipage}
    \end{learningStandardBox}
}


\newcommand{\exampleBox}[2]{%
    \begin{tcolorbox}[title=\texttt{Example}]
        \texttt{Problem:} {#1} 
        \texttt{Solution:} {#2}
    \end{tcolorbox}%
}

\newcommand{\customsection}[1]{%
    \vspace{1em} % Add some vertical space above (optional)
    {\Large\bfseries #1} % Format the title: large font and bold
    \par\vspace{0.5em}\noindent % Add some space below
}