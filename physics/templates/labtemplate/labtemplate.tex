
\input{labpreamble.tex}
%%%%%%%%%%%%%%%%%%%%%% BEGIN DOCUMENT %%%%%%%%%%%%%%%%%%%%%%%%%%%%%%%%%%%%%%%
\begin{document}

% Place name, section, and instructor details at the top left
\vspace*{-0.2in} % Adjust this to control vertical spacing from the top of the page
\noindent
\makebox[0.65\textwidth][l]{Name:\enspace\hrulefill\,\,\,} 
\makebox[0.1\textwidth][l]{\,Period:\enspace} 
\makebox[0.25\textwidth][l]{\hrulefill}\\[0.25cm]
\makebox[0.65\textwidth][l]{Instructor: \enspace\texttt{\underline\instructor}}\hfill 
\makebox[0.1\textwidth][l]{ Course: }\hfill
\makebox[0.25\textwidth][l]{ \texttt{\underline\coursename}}\hfill \\[0.25cm]
\makebox[0.65\textwidth]{}\makebox[0.1\textwidth][l]{ Term: }\hfill
\makebox[0.25\textwidth][l]{\enspace\texttt{\underline{\term\ \courseyear}}}\hfill 
\hfill{Score:\underline{\hspace{2cm}}/\numpoints}
\vspace{0.5in} % Space before the title

\begin{centering}
\noindent\textbf{\Large \worksheetname} \\[0.1in]
\end{centering}
\vspace{0.1in}
\epigraph{\itshape``If you cannot measure it, you cannot improve it.''}{---Lord Kelvin}
\qsp
% Math is the language of physics. Although the scope of this course will be largely \emph{qualitative} (\emph{i.e.} conceptual), we will need to have a certain level of mathematical fluency under our belts in order to describe the universe around us. Physics is by nature a \emph{quantitative} science--- in other words: numbers are necessary to accurately and precisely depict the world around us. 
\textbf{\Large Objective:}

\qsp
Estimate the number of softballs that can fit in the classroom.
\qsp

\textbf{\Large Pre-lab Question}
\qsp
\begin{questions}

\question[1] Before taking any measurements, make a guess as to how many softballs you think would fit in the classroom:
\qsp
\begin{align*}
N_{guess} =& \underline{\hspace{2.5cm}}
\end{align*}

\textbf{\Large Materials}
\qsp
\begin{itemize}
\item A softball
\item A meterstick
\item A reel tape measure
\end{itemize}
\qsp
\textbf{\Large Procedure}
\qsp
\question[2] Measure the volume of the ball. Recall that the volume $V$ of a sphere of radius $r$ is given by 
\begin{align*}
V_{sphere} =& \frac{4}{3}\pi r^3.
\end{align*}
 What is your measurement of the radius $r$ of the ball? Write your answer on the line below.
\qsp

\begin{align*}
r_{ball} =& \underline{\hspace{2.5cm}}\SI{}{cm}
\end{align*} 

What is your calculated volume of the ball? 
\qsp 

\begin{align*}
V_{ball} =& \underline{\hspace{2.5cm}}\SI{}{cm^3}
\end{align*} 

\question[2] Measure the volume of the classroom. Recall that the volume of a rectangular prism with width $w$, length $l$, and height $h$ has a volume of

\begin{align*}
V_{rect} = w\times l\times h.
\end{align*}

What is your measurement of the width of the room? Write your answer on the line below.
\qsp

\begin{align*}
w_{room} =& \underline{\hspace{2.5cm}}\SI{}{cm}
\end{align*} 

What is your measurement of the length of the room? 
\qsp

\begin{align*}
l_{room} =& \underline{\hspace{2.5cm}}\SI{}{cm}
\end{align*} 

What is your measurement of the height of the room? 
\qsp

\begin{align*}
h_{room} =& \underline{\hspace{2.5cm}}\SI{}{cm}
\end{align*} 

What is your calculated volume of the classroom? 
\qsp 

\begin{align*}
V_{room} =& \underline{\hspace{2.5cm}}\SI{}{cm^3}
\end{align*} 
\question[3] What about the other objects in the room (tables, equipment, etc.)? How would you account for the volume of these objects in your calculations? 
\begin{parts}
\part Use the space below to list each type of object, the volume of the object, and the number of those objects. 
\qsppp
\part By multiplying each object type by the number of that object, add up the total volume of the objects:
\qsp
\begin{align*}
V_{objects} =& \underline{\hspace{2.5cm}}\SI{}{cm^3}
\end{align*} 
\qsp
\end{parts}
\question[2] Finally, devise a method incorporating your calculated volumes $V_{ball}$, $V_{room}$, and $V_{objects}$ to calculate the number of softballs that would fit in the classroom along with the objects currently inside of it. 
\begin{parts}
\part Describe your reasoning for your calculation and outline your calculation of the number of softballs $N$.
\qsppp
\part What is your final calculated estimate of $N$?
\qsp
\begin{align*}
N =& \underline{\hspace{2.5cm}}
\end{align*}
\end{parts}

\end{questions}
\end{document}
