
\documentclass[11pt,addpoints,letterpaper]{exam}
\usepackage{adjustbox} % Add this to the preamble
\usepackage{float}

\usepackage{amsmath,tcolorbox,mdframed,xcolor}
\usepackage[top=0.75in, bottom=0.85in, left=1in, right=1in]{geometry} % Adjust margins
\usepackage{mathtools}
\usepackage{siunitx}
\usepackage{nccmath}
\usepackage{physics}
\usepackage{tikz}
\usepackage{multicol}
\usepackage{epigraph}
\usepackage{cancel}
\usepackage{xfrac}
\usepackage{pgfplots}
\usepackage{array}   % For alignment
\usepackage{booktabs} % For better horizontal lines
\usepackage{caption} % For caption formatting
\usepackage{graphicx} % For scaling tables if needed
\usepackage{setspace}

\setlength\epigraphwidth{.8\textwidth}
\setlength\epigraphrule{0pt}

\noprintanswers

\newcommand{\Section}[1]{
\textbf{\\\Large #1\\}
}
\newcommand{\labheader}[4]{%
  \vspace*{-0.2in} % Adjust vertical spacing
  \noindent
  \makebox[0.65\textwidth][l]{Name:\enspace\hrulefill\,\,\,}
  \makebox[0.1\textwidth][l]{\,Period:\enspace}
  \makebox[0.25\textwidth][l]{\hrulefill}\\[0.25cm]
  \makebox[0.65\textwidth][l]{Instructor: \enspace\texttt{\underline{#1}}}\hfill
  \makebox[0.1\textwidth][l]{ Course: }\hfill 
  \makebox[0.25\textwidth][l]{ \texttt{\underline{#2}}}\hfill \\[0.25cm]
  \makebox[0.65\textwidth]{}\makebox[0.1\textwidth][l]{ Term: }\hfill
  \makebox[0.25\textwidth][l]{\enspace\texttt{\underline{#3}}}\hfill
  % \hfill{Score: \underline{\hspace{2cm}}/#4}
  \vspace{0.5in}
}

% Define variables
\newcommand{\instructor}{Mr. Rodriguez}
\newcommand{\coursename}{Conceptual Physics A}
\newcommand{\term}{Winter}
\newcommand{\courseyear}{2024-2025} % Renamed \year to \courseyear to avoid conflict
\newcommand{\instructions}{}
\newcommand{\worksheetname}{Lab 4: Inertia and Oscillations}
\newcommand{\qspppp}{\vspace{6cm}}
\newcommand{\qsppp}{\vspace{4.5cm}}
\newcommand{\qspp}{\vspace{2.5cm}}
\newcommand{\qsp}{\vspace{0.6cm}}

\tcbset{
    myboxstyle/.style={
        colback=white,        % Background color
        colframe=black,       % Border color
        boxrule=0.5pt,        % Border thickness
        arc=5mm,              % Rounding radius
        boxsep=2mm,           % Padding around text
        left=4pt, right=4pt,  % Inner padding on left and right
    }
}
\newcommand{\equationbox}[2]{
\begin{center}
\begin{tcolorbox}[colframe=black!60, colback=white, arc=5mm, boxrule=0.75pt, title=#1]
% \vspace{-0.2in} % Tightens the space between the title and content
\begin{flushleft}
\begin{align}
#2
\end{align}
\end{flushleft}
\end{tcolorbox}
\end{center}
}
\newcommand{\equationboxx}[2]{
\begin{center}
\begin{tcolorbox}[colframe=black!60, colback=white, arc=5mm, boxrule=0.75pt, title=#1]

#2

\end{tcolorbox}
\end{center}
}
\newcommand{\instructionbox}[1]{\begin{center}\fbox{\fbox{{\centering #1}}}\end{center}}

\newcommand{\learningStandard}[1]{
    \small
    \tcbox[colback=gray!10, colframe=black, boxrule=0.5mm, sharp corners=southwest]{
        \textbf{#1} \hspace{0.5cm} % Spacing between title and vertical rule
        \textcolor{gray}{\vrule height 10pt width 0.3pt}\hspace{0.5cm} % Lighter and thinner vertical rule
        \textbf{Score:} \rule{1.5cm}{0.5pt} /10
    }
}

\newtcolorbox{learningStandardBox}[2]{%
    title={Learning Standard #1}, % Title with manual input for the standard number
    fonttitle=\bfseries\small,  % Title font style
    colback=gray!15, % Background color for main content
    colframe=black, % Frame color
    coltitle=black!80, % Font color for title
    colbacktitle=gray!60, % Background color for title
    boxrule=0.5mm, % Border thickness
    sharp corners=south, % Rounded top corners only
    width=\textwidth, % Full text width
    halign=center, % Center-align content
    top=10pt, % Adds extra space at the top to move content down
}
\newcommand{\standardBox}[2]{%
    \begin{learningStandardBox}{#1}{#2} % Passes number and name as arguments
        \begin{minipage}[t]{0.5\textwidth} % Left section for standard name
            \textit{#2} % Italicized name of the standard
        \end{minipage}%
        \hfill
        \begin{minipage}[t]{0.25\textwidth} % Middle section for score
            \textbf{Score:} \rule{1.5cm}{0.5pt} /10
        \end{minipage}%
        \hfill
        \begin{minipage}[t]{0.2\textwidth} % Right section for grade
            \textbf{Grade:} \phantom{1.2cm}
        \end{minipage}
    \end{learningStandardBox}
}
\newcommand{\standardBoxnoscore}[2]{%
    \begin{learningStandardBox}{#1}{#2} % Passes number and name as arguments
        \begin{minipage}[t]{0.5\textwidth} % Left section for standard name
            \textit{#2} % Italicized name of the standard
        \end{minipage}%
        \hfill
        \begin{minipage}[t]{0.25\textwidth} % Middle section for score
            \textbf{Score:} \rule{1.5cm}{0.5pt} /0
        \end{minipage}%
        \hfill
        \begin{minipage}[t]{0.2\textwidth} % Right section for grade
            \textbf{Grade: N/A} \phantom{1.2cm}
        \end{minipage}
    \end{learningStandardBox}
}


\newcommand{\exampleBox}[2]{%
    \begin{tcolorbox}[title=\texttt{Example}]
        \texttt{Problem:} {#1\\} 
        \texttt{Solution:} {#2}
    \end{tcolorbox}%
}

\newcommand{\customsection}[1]{%
    \vspace{1em} % Add some vertical space above (optional)
    {\Large\bfseries #1} % Format the title: large font and bold
    \par\vspace{0.5em}\noindent % Add some space below
}

%%%%%%%%%%%%%%%%%%%%%% BEGIN DOCUMENT %%%%%%%%%%%%%%%%%%%%%%%%%%%%%%%%%%%%%%%
\begin{document}

% Place name, section, and instructor details at the top left
\vspace*{-0.2in} % Adjust this to control vertical spacing from the top of the page
\noindent
\makebox[0.65\textwidth][l]{Name:\enspace\hrulefill\,\,\,} 
\makebox[0.1\textwidth][l]{\,Period:\enspace} 
\makebox[0.25\textwidth][l]{\hrulefill}\\[0.25cm]
\makebox[0.65\textwidth][l]{Instructor: \enspace\texttt{\underline\instructor}}\hfill 
\makebox[0.1\textwidth][l]{ Course: }\hfill
\makebox[0.25\textwidth][l]{ \texttt{\underline\coursename}}\hfill \\[0.25cm]
\makebox[0.65\textwidth]{}\makebox[0.1\textwidth][l]{ Term: }\hfill
\makebox[0.25\textwidth][l]{\enspace\texttt{\underline{\term\ \courseyear}}}\hfill 
\hfill{Score:\underline{\hspace{2cm}}/\numpoints}
\vspace{0.5in} % Space before the title

\begin{centering}
\noindent\textbf{\Large \worksheetname} \\[0.1in]
\end{centering}
\vspace{0.1in}
\epigraph{\itshape``If you cannot measure it, you cannot improve it.''}{---Lord Kelvin}
\qsp
% Math is the language of physics. Although the scope of this course will be largely \emph{qualitative} (\emph{i.e.} conceptual), we will need to have a certain level of mathematical fluency under our belts in order to describe the universe around us. Physics is by nature a \emph{quantitative} science--- in other words: numbers are necessary to accurately and precisely depict the world around us. 
\textbf{\Large Objective:}

\qsp
Estimate the number of softballs that can fit in the classroom.
\qsp

\textbf{\Large Pre-lab Question}
\qsp
\begin{questions}

\question[1] Before taking any measurements, make a guess as to how many softballs you think would fit in the classroom:
\qsp
\begin{align*}
N_{guess} =& \underline{\hspace{2.5cm}}
\end{align*}

\textbf{\Large Materials}
\qsp
\begin{itemize}
\item A softball
\item A meterstick
\item A reel tape measure
\end{itemize}
\qsp
\textbf{\Large Procedure}
\qsp
\question[2] Measure the volume of the ball. Recall that the volume $V$ of a sphere of radius $r$ is given by 
\begin{align*}
V_{sphere} =& \frac{4}{3}\pi r^3.
\end{align*}
 What is your measurement of the radius $r$ of the ball? Write your answer on the line below.
\qsp

\begin{align*}
r_{ball} =& \underline{\hspace{2.5cm}}\SI{}{cm}
\end{align*} 

What is your calculated volume of the ball? 
\qsp 

\begin{align*}
V_{ball} =& \underline{\hspace{2.5cm}}\SI{}{cm^3}
\end{align*} 

\question[2] Measure the volume of the classroom. Recall that the volume of a rectangular prism with width $w$, length $l$, and height $h$ has a volume of

\begin{align*}
V_{rect} = w\times l\times h.
\end{align*}

What is your measurement of the width of the room? Write your answer on the line below.
\qsp

\begin{align*}
w_{room} =& \underline{\hspace{2.5cm}}\SI{}{cm}
\end{align*} 

What is your measurement of the length of the room? 
\qsp

\begin{align*}
l_{room} =& \underline{\hspace{2.5cm}}\SI{}{cm}
\end{align*} 

What is your measurement of the height of the room? 
\qsp

\begin{align*}
h_{room} =& \underline{\hspace{2.5cm}}\SI{}{cm}
\end{align*} 

What is your calculated volume of the classroom? 
\qsp 

\begin{align*}
V_{room} =& \underline{\hspace{2.5cm}}\SI{}{cm^3}
\end{align*} 
\question[3] What about the other objects in the room (tables, equipment, etc.)? How would you account for the volume of these objects in your calculations? 
\begin{parts}
\part Use the space below to list each type of object, the volume of the object, and the number of those objects. 
\qsppp
\part By multiplying each object type by the number of that object, add up the total volume of the objects:
\qsp
\begin{align*}
V_{objects} =& \underline{\hspace{2.5cm}}\SI{}{cm^3}
\end{align*} 
\qsp
\end{parts}
\question[2] Finally, devise a method incorporating your calculated volumes $V_{ball}$, $V_{room}$, and $V_{objects}$ to calculate the number of softballs that would fit in the classroom along with the objects currently inside of it. 
\begin{parts}
\part Describe your reasoning for your calculation and outline your calculation of the number of softballs $N$.
\qsppp
\part What is your final calculated estimate of $N$?
\qsp
\begin{align*}
N =& \underline{\hspace{2.5cm}}
\end{align*}
\end{parts}

\end{questions}
\end{document}
